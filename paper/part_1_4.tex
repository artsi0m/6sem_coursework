\subsection{Определение параметров воздействующих дестабилизирующих факторов протекающих физических процессов}


РЭС эксплуатируются в условиях воздействия на них целого ряда
систематических и случайных факторов. Систематические факторы определяют рабочие функции аппаратуры и могут быть учтены в процессе разработки,
а случайные факторы по своему характеру и времени воздействия проявляются случайным образом, в связи с чем учет их влияния в процессе разработки достаточно затруднен ~\cite{Alexeev2013}.

Факторы делятся на объективные, не зависящие от человека, и субъективными, то есть зависящими от действий конструктора или использующего данный тестовый стенд оператора.

По характеру воздействия на РЭС объективные факторы разделяются на климатические, механические и другие.

Параметры климатических факторов принимают значения в соотвестиве с учетом УХЛ 4.2 ГОСТ 15150-69.

Параметры механических факторов же принимают значения, выбранные разработчиком.

Наиболее вероятным механическим фактором для данного устройства будет воздействие вибрации, в следствии работы подключаемых сервоприводов.

Во всей схеме практически не встречаются компоненты с большим тепловыделением. Так например найти коэффициент \textit{TDP} для микроконтроллера не предстовляется возомжным, потому что среди производителй микроконтроллеров, в отличие от производителей микропроцессоров не принято не то что рассчитывать этот коэффициент рассеивания теплоты, а даже закладывать в этот компоенент схемы возможности какого-либо его нагрева.

Однако это ни в коем случае не означает то, что ни один из элементов не будет нагреваться. Напротив при работе любого электронного прибора какая-то часть потребляемой мощности рассеивается как тепло.
Но если рассматривать отношение мощности потребляемой и рассеиваемой как тепло, то можно понять, что самым простым оно будет у резисторов.

По видимому, можно взять информацию о мощности резистора и принять её за мощность рассеиваемую в виде теплоты ~\cite{HeatDissipatedResistors}.

Примерно таким же образом было сделано допущение о схожести в вопросе рассеиваемой мощности между резисторами и потенциометрами.

Исходя из всего вышеописанного был сделан вывод о необходимости проведения частотного анализа и термического анализа.

\newpage