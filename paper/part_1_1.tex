\section{ОБЩЕТЕХНИЧЕСКИЙ АНАЛИЗ ПРОЕКТИРУЕМОГО УСТРОЙСТВА}
\subsection{Анализ исходных данных}

В курсовой работе рассматривается система тестирования сервоприводов квадрокоптера.
Назначение этой системы — тестирование отдельных частей квадрокоптера,
таких как сервоприводы,
блок управления скоростью или контроллер полёта ~\cite{Elector521}.
Говоря кратко — это тестовый стенд.


Тестовый стенд выполнен в виде платы,
на которую с помощью монтажа в отверстия помещаются компоненты.
Вероятно, что такой способ мантажа выбран для того,
чтобы облегчить сборку данной схемы,
установку и смену тестируемых компонентов
или незначительного изменения схемы,
необходимого для взаиомодействия с определенными компонентами.


Поскольку в курсовой работе будет производиться расчёт тепловых
режимов будет также важно учитывать климатические факторы внешней
среды, то какие условия эксплутации при этом должны быть соблюдены
регламентирует соответсвующий стандарт — ГОСТ 15150-69
~\cite{GOST-15150-69}.

Настоящий стандарт должен применяться при проектировании изделий.  В
частности, он должен применяться при составлении технических заданий
на разработку или модернизацию изделий, а также при разработке
государственных стандартов и технических условий, устанавливающих
требования в части воздействия климатических факторов внешней среды
для группы изделий, а при отсуствии указанных групповых документов —
для отдельных видов изделий ~\cite{GOST-15150-69}.

Для конкретных типов или групп изделий виды воздействующих
климатических факторов и их номинальные значения устанавливаются в
зависимости от условий эксплуатации изделий в соответвующих
технических заданих, стандартах и технических условиях ~\cite{GOST-15150-69}.


Так как в данном случае тестовый стенд используется для того,
чтобы тестировать отдельные компоненты и их взаимодейтевие,
то логично предположить,
что в течении всего жизненного цикла изделия,
оно не будет покидать пределов лаборатории,
в которой будут проводиться испытания,
осуществляемые на данном тестовом стенде.
Основываясь на этом, можно говорить,
что климатические условия эксплуатации данного устройства соответвуют категории УХЛ 4.2 ГОСТ 15150-69.


Характеристика данной категории следующая:\\
Для эксплуатации в помещниях (объемах) с искуственно регулируемыми
климатическими условиями, например в закрытых отапливаемых или
охлаждаемых и вентелируемых производственных и других, в том числе
хорошо вентилиуруемых подземных помещниях (отсуствие воздействия
прямого соленчного излучения, атмосферных осадков, ветра, песка и пыли
наружного воздуха; отсутвие или существенное уменьшение воздействия
рассеяного солнечного излучения и конденсации влаги). Для эксплуатации
в лабораторных, капитальных и других подобного типа помещениях ~\cite{GOST-15150-69}.

Соответсвие условий работы тестового стенда определенной категории,
указанной в ГОСТ, важно по той причине,
что расчеты внешних тепловых воздействией оказываемых на данную РЭС будут производиться с использованием исходных данных о температуре,
указанных в конкретной категории ГОСТ.