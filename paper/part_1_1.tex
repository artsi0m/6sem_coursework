\section{ОБЩЕТЕХНИЧЕСКИЙ АНАЛИЗ ПРОЕКТИРУЕМОГО УСТРОЙСТВА}
\subsection{Анализ исходных данных}

В курсовой работе рассматривается система тестирования сервоприводов квадрокоптера.
Назначение этой системы — тестирование отдельных частей квадрокоптера,
таких как сервоприводы,
блок управления скоростью или контроллер полёта ~\cite{Elector521}.
Говоря кратко — это тестовый стенд.


Тестовый стенд выполнен в виде платы,
на которую с помощью монтажа в отверстия помещаются компоненты.
Вероятно, что такой способ мантажа выбран для того,
чтобы облегчить сборку данной схемы,
установку и смену тестируемых компонентов
или незначительного изменения схемы,
необходимого для взаиомодействия с определенными компонентами.


Поскольку в курсовой работе будет производиться расчёт тепловых
режимов будет также важно учитывать климатические факторы внешней
среды, то какие условия эксплутации при этом должны быть соблюдены
регламентирует соответсвующий стандарт — ГОСТ 15150-69
~\cite{GOST-15150-69}.

Настоящий стандарт должен применяться при проектировании изделий.  В
частности, он должен применяться при составлении технических заданий
на разработку или модернизацию изделий, а также при разработке
государственных стандартов и технических условий, устанавливающих
требования в части воздействия климатических факторов внешней среды
для группы изделий, а при отсуствии указанных групповых документов —
для отдельных видов изделий ~\cite{GOST-15150-69}.

Для конкретных типов или групп изделий виды воздействующих
климатических факторов и их номинальные значения устанавливаются в
зависимости от условий эксплуатации изделий в соответвующих
технических заданих, стандартах и технических условиях ~\cite{GOST-15150-69}.


Так как в данном случае тестовый стенд используется для того,
чтобы тестировать отдельные компоненты и их взаимодейтевие,
то логично предположить,
что в течении всего жизненного цикла изделия,
оно не будет покидать пределов лаборатории,
в которой будут проводиться испытания,
осуществляемые на данном тестовом стенде.
Основываясь на этом, можно говорить,
что климатические условия эксплуатации данного устройства соответвуют категории УХЛ 4.2 ГОСТ 15150-69.


Характеристика данной категории следующая:\\
Для эксплуатации в помещниях (объемах) с искуственно регулируемыми
климатическими условиями, например в закрытых отапливаемых или
охлаждаемых и вентелируемых производственных и других, в том числе
хорошо вентилиуруемых подземных помещниях (отсуствие воздействия
прямого соленчного излучения, атмосферных осадков, ветра, песка и пыли
наружного воздуха; отсутвие или существенное уменьшение воздействия
рассеяного солнечного излучения и конденсации влаги). Для эксплуатации
в лабораторных, капитальных и других подобного типа помещениях ~\cite{GOST-15150-69}.

Соответсвие условий работы тестового стенда определенной категории,
указанной в ГОСТ, важно по той причине,
что расчеты внешних тепловых воздействией оказываемых на данную РЭС будут производиться с использованием исходных данных о температуре,
указанных в конкретной категории ГОСТ.


Также исходя из правил установленных ГОСТ ЕСКД устройству назначен код из классификатора.
При этом использовался общероссийский классификатор изделий и конструкторских документов ОК 012-93.
Этот код ГУИР 467993.013. Здесь ГУИР это код предприятия, а 013 порядковый регистрационный номер,
соответствующий номеру студента в списке группы.
Остальные шесть цифр обозначают то чем именно является устройство.
Первые две цифры — 46 — класс устройства, последующие четыре цифры,
это подкласс, группа, подгруппа и вид устройства.

Класс устройcтва 46 обозначает средства радиоэлектронного управления, связи, навигации и вычислительной техники.
Это довольно обширный класс, под который подпадает большинство возможных устройств,
которые могут быть разработаны инженером по радиоэлектронике.

Подкласс и группа относит устройство к комплектам монтажным и эксплуатационным,
имитаторам,
средствам контрольно-испытательным.

Подгруппа и вид устройства обозначают устройство как электронное средство контрольно-испытательных составных частей функциональных.

Исследование методом конечных элементов
позволяет сократить число изготавливаемых прототипов и проводимых эксперементов с ними,
при конструировании, разработке и контроле за устройством или процессом.
Это не означает, что компании или исследовательские институты,
применяющие исследования методом конечных элементов экономят деньги применяя МКЭ.
Однако за ту же цену они получают больше разработок,
что позволяет им иметь преимущество при конкуренции.
По этой причине может быть смысл в том,
чтобы увеличить ресурсы затарчиваемые на исследования и разработку для МКЭ.
Когда МКЭ модель создаётся и становится полезной она может привнести
понимание о небходимых для совершенствования устройства изменений,
которое сложно получить пользуясь одной только интуицией~\cite{MULTIPHYSICS-CYCLOPEDIA-FEM}.


Обосновав важность моделирования,
можно перечислить программы используемые для его осущствления в этом курсовом.
Это система автоматизированного проектирования \textit{SOLIDWORKS}.
Одно из дополнений (\textit{Add-Ins}) используемое для симцуляций в этой программе — \textit{SOLIDWORKS Simulation}.
Также использовались такие программы как \textit{COMSOL Multiphysics} и \textit{Ansys Workbench}.

\textit{COMSOL Multiphysics} и \textit{Ansys Workbench} это программы относящиеся к классу \textit{CAE} программ.
Аббревитура \textit{CAE} рассшифровывается как \textit{computer aided engineering},
переводится как система автоматизации инженерных расчетов и анализа и обозначет целый класс программ,
предназначенных для инженерных рассчетов.

При этом в русском языке существует общеупотребимый термин САПР — системы автоматизированного проектирования,
но он настолько всеобъемлющ, что включает в себя как программы для инженерных рассчетов используемые в данной работе,
так и просто программы для создания чертежей и трёхмерных моделей детали или сборки.

Если же уточнять, а не обобщать, то можно сказать,
что и \textit{COMSOL Multiphysics}, и \textit{SOLIDWORKS Simulation},  и \textit{ANSYS Workbench},
являются программным обеспечением для вычислений методом конечных элементов.

Проведение анализа сразу в нескольких \textit{CAE}-программах позволяет сравнить результаты и проверить адекватность полученных данных.