\tableofcontents
\newpage
\begin{center}
\textbf{Перечень условных обозначений, символов и терминов}
\end{center}

ГОСТ — государственный стандарт.

МКЭ – метод конечных элементов.

САПР — системы автоматизированного проектирования.

РЭС — радиоэлектронное средство.

ШИМ — широко импульсная модуляция.

ПП — печатная плата.

\textit{CAE} — \textit{Computer Aided Engineering}, система автоматизации инженерынх расчетов.

\textit{DPDT} — \textit{Double Pole Double Throw}, переключатель два полюса, два направления.

\textit{GPIO} — \textit{ General Purpose Input Output}, система ввода-вывода общего пользования

\textit{OLED} — \textit{Organic light emitted diode}, органический светодиод.

\textit{I2C} — \textit{Inter-Integrated Circuit}, интерфейс микрокнтроллера.

\textit{TDP} — \textit{Thermal Dessipation Power}, рассеиваемая тепловая мощность.

\textit{PCB} — \textit{Printed Circuti Board}, печатная плата.

\newpage

\begin{center}
\textbf{ВВЕДЕНИЕ}
\end{center}

Владение системами автоматизированных инженерных расчетов явлюятся одним из важным навыков современного инженера.

В данной курсовой работе рассмотрены результаты моделирования печатной платы и анализа несколькими системами автоматизированных инженерных расчетов, использующих метод конечных элементов.

Проведен частотный и термический анализ.

Частотный анализ позволяет оценить физические характеристики печатной платы на различных рабочих частотах, что необходимо для обеспечения
стабильной работы устройства в условиях переменных нагрузок и окружающей среды.

Термический анализ, в свою очередь, позволяет выявить и устранить потенциальные проблемы с перегревом компонентов, что является критическим
аспектом для обеспечения долговечности и надежности работы устройства.
Целью курсовой работы является общетехнический анализ устройства,
моделирование его физических процессов, а также анализ полученных результатов.

\newpage