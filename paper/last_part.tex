\begin{center}
\textbf{Заключение}
\end{center}

В результате выполнения курсового проекта проведен общетехнический
анализ проектируемого устройства, который включает: анализ исходных данных, описание принципа работы анализируемого устройства, анализ электрической принципиальной схемы устройства, определение параметров воздействующих дестабилизирующих факторов и протекающих физических процес-
лсов для последующего моделирования.

Проведено моделирование физических процессов, воздействующих на
устройство, а именно: обоснование выбора прикладного программного обеспечения для моделирования физических процессов, разработан план моделирования физических процессов, разработана методика построения трехмерной модели исследуемого устройства, проведено моделирование механических
процессов, протекающих в электронном модуле и устройстве в целом, и проведено моделирование тепловых процессов, протекающих в электронном модуле и устройстве в целом.
Основываясь на вышеперечисленном, был проведен анализ полученных
результатов, а также дана краткая оценка.

Таким образом была предпринята попытка реализовать все поставленные цели и задачи на курсовой проект. Однако с уверенностью можно заключить, что эта попытка не была удачной. Большую роль сыграли проблемы с экспортом модели трехмерной модели, случившиеся в результате перехода на другую систему автоматизированного проектирования печатной платы.

Остается только сделать выводы и не допускать ошибок, подобных допущенным при выполнении данного курсового проекта.
  \newpage



% \bibligoraphystyle{5sem}
% \renewcommand{\bibsection}{{Cписок использованных источников}}
\renewcommand{\refname}{\textbf{Cписок использованных источников}}
\DeclareFieldFormat{url}{Режим доступа\addcolon\space\url{#1}}
\DeclareFieldFormat{title}{{#1}}
\DeclareFieldFormat{labelnumberwidth}{[{#1}]\adddot  }
\printbibliography
