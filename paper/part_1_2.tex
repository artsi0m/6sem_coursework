\subsection{Описание принципа работы анализируемого устройства}

Устройство представляет собой печатную плату, с установленными на ней навесным монтажом компонентами.


Сердце цепи — микроконтроллер \textit{ATmega328P},
который известен тем, что также используется в плате \textit{ArduinoUNO}.
Четыре пина \textit{GPIO} этого микрокнтроллера подключены к выводам серво-сигнала.
Серво-сигналом здесь назвам ШИМ сигнал с частотой 50 Гц.
Ширина импульса, время когда сигнал высок,
должно быть в промежутке с одной до двух милисекунд,
остальное время сигнал низок. При ширине импульса в 1,5 милисекунды сервопривод в промежуточной позиции и не вращается.
Если импульс короче, то сервопривод поворачивается в одно направление, если длиннее,
он поворачивается в противположное направление ~\cite{Elector521}.


Малогабаритные быстродействующие сервоприводы применяются в современных высокоточных системах управления подвижными объектами: рулевыми системами летательных аппаратов, автоматическими манипуляторами, роботами с подвижными элементами конструкции и др~\cite{dyakovSUBSTANTIATIONRELIABILITYSERVOMOTORS2023}.


В типичной системе контроля полета квадракоптера или дрона
пять парамеров подлежат отслеживанию:
длина четырех управляющих сервоприводами импульсов,
которые контролируют тягу, крен, тангаж и рысканье,
и напряжение питания.
В случае сложной конструкции,
например при разработке полётного контроллера,
данный тестовый стенд позволяет наблюдать получаемый сигнал ~\cite{Elector521}.



Данный тестовый стенд может работать в двух режимах,
выбираемых переключателем:
\begin{enumerate}[label={\arabic*.}]
\item Ручной режим. В нём тестовый стенд генерирует импульсы для четырех сервоприводов или полетного контроллера. Длина импульсов контролируется четырьмя потенциометрами. В этом режиме тестовый стенд обеспечивает питание сервоприводов или полетного контроллера и питание от квадрокоптера не должно быть подключено к ним. Напряжение питания тестового стенда должно быть между 7,5 или 12 Вольт.
\item Режим ввода. В нём измеряются длины имульсов поступающих от приёмника сигнала. Сигналы потом поступают на выводы, подключенные к полётному контроллеру или сервпоприводам.
  В этом режиме тестовый стенд и сервоприводы получают питание от источника питания от квадрокоптера. Получаемое питание не должно превышать 7.49 Вольт и тестовый стенд не должен быть подключен к своему истчонику питания. Также четыре канада должны быть подключены, иначе светодиод и зуммер просигналируют об ошибке.  
\end{enumerate}


Тестовый стенд измеряет длину контрольных импульсов и даёт информацию о качестве источника питания.
Данное устройство может быть подключено между приёмником дистанционного управления и полетным контроллером дрона,
или между полетным контроллером и сервоприводами.
Информацию о сигналах выводимых к сервоприводам тестовый стенд выводит на \textit{OLED} дисплей подключенный, через \textit{I2C} интерфейс.
Дисплей показывает продолжительсность импульсов графиком из четырёх кривых вмести с их численным значением в микросекундах ~\cite{Elector521}.