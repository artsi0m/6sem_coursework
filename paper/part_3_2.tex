\subsection{Обработка, анализ и интепретация данных проведенного моделирования телповых процессов}

\begin{table}[h]
  \centering
  \caption{Результаты теплового анализа в \textit{SolidWorks}}
\begin{tabular}{|c|c|}
\hline
  Вариант компоновки ПП & Максимальное начение температуры в Цельсиях \\
  1 &  150 \\
  2 &  145 \\
  3 &  143 \\
\hline
\end{tabular}
\end{table}

Самое большое значение температуры у печатной платы в первой компановке.
По видимому оба решения с разнесеним компонентов по разные стороны платы
и заменой тепловыделяющих компонентов на меньшие по размеру выглядят приемлимыми для обеспечения теплового режима ПП.

Но только теплового режима  печатной платы.

\begin{table}[h]
  \centering
  \caption{Результаты теплового анализа в \textit{COMSOL Multiphysics}}
\begin{tabular}{|c|c|}
\hline
  Вариант компоновки ПП & Значение максимальной температуры в Цельсиях \\
  1 & 41.3 \\
  2 & 47,5 \\
  3 & 47 \\
\hline
\end{tabular}
\end{table}

Моделирование в \textit{COMSOL Multiphysics} показало ровно противоположный результат. При этом в качестве геометрии была импортирована модель из \textit{SOLIDWORKS} с расширением \textit{.SLPDRT}

По видимому несовпадение результатов вызвано различным методом решения применямым в данных программах.

Однако я склонен больше доверять \textit{COMSOL}, как программер специально предназначенных для таких рассчетов.

По полученным результатам и графикам можно сделать вывод,
что при проектировании печатной платы, точнее при, так называемой разводке, решение о том как расставлять компоненты можно принимать
не только основываясь на том как близко должен быть элемент с точки зрения схемотехники, но также учитывать, что рассеивающие тепло в значительной степени компоненты будут нагреваться до больших температур в тех случаях, когда будут расставлены рядом.

\newpage