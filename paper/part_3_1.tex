\newpage
\section{Анализ полученных результатов моделирования}
\subsection{Обработка, анализ и интепретация данных проведенного моделирования механических процессов}

\begin{table}[h]
  \centering
  \caption{Результаты частотного анализа в \textit{SolidWorks}}
\begin{tabular}{|c|c|}
\hline
  Вариант компоновки ПП & Значение собственной частоты \\
  1 & 10,1Гц \\
  2 & 7,3 Гц \\
  3 & 7,9 Гц \\
\hline
\end{tabular}
\end{table}

Самое большое значение частоты у печатной платы выполненненой в первой компоновке. Однако можно заметить, что во всех трёх случаях частота досточно мала. 10 Гц это 10 колебаний платы в секунду, что несоотвевует уровню требуему данной печатной плате.

Однако можно считать, что в значительной степени такой результат вышел потому, что плата крепилась только за одну поверхность и была очень тонкой, так как такой вышла модель в результате экспорта.

\begin{table}[h]
  \centering
  \caption{Результаты частотного анализа в \textit{COMSOL Multiphysics}}
\begin{tabular}{|c|c|}
\hline
  Вариант компоновки ПП & Значение собственной частоты \\
  1 & 0.1 Гц \\
  2 & 13509 Гц \\
  3 & 14 Гц \\
\hline
\end{tabular}
\end{table}

Результаты анализа в \textit{COMSOL Multiphysics} так разнятся, что сделать какой-либо вывод, а не исключить данные из анализа сложно.
Но судя по всем именно значение 13509 Гц ближе всего к тому значению, которое должно было получиться при анализе печатной платы в независомости от компоновки.
