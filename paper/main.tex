%%% Local Variables: 
%%% coding: utf-8
%%% mode: latex
%%% TeX-engine: xetex
%%% End:

% main.tex файл который компилируется и подключает за собой остальный
% файлы с содержимым через input
% При этом согласно моей же логике преамбула должна отстаться здесь.

% Используем опцию emptystyle т.к. bsuir-std расширяет eskdxtext,
% который без этой опции сделает рамку для чертежа даже на документе
% с просто текстом.
\documentclass[a4paper,emptystyle]{bsuir-std}


\usepackage[sorting=none,backend=biber]{biblatex}
% библиогорафия в last
\usepackage{url}

\usepackage{float}



\addbibresource{Karakin6semCoursework.bib}

\begin{document}





% Путь к месту где картинки лежат
\graphicspath{ {..} }

%%% Титульник
\input{title}


%%% Введение.
\tableofcontents
\newpage
\begin{center}
\textbf{Перечень условных обозначений, символов и терминов}
\end{center}

ГОСТ — государственный стандарт.

МКЭ – метод конечных элементов.

САПР — системы автоматизированного проектирования.

РЭС — радиоэлектронное средство.

ШИМ — широко импульсная модуляция.

ПП — печатная плата.

\textit{CAE} — \textit{Computer Aided Engineering}, система автоматизации инженерынх расчетов.

\textit{DPDT} — \textit{Double Pole Double Throw}, переключатель два полюса, два направления.

\textit{GPIO} — \textit{ General Purpose Input Output}, система ввода-вывода общего пользования

\textit{OLED} — \textit{Organic light emitted diode}, органический светодиод.

\textit{I2C} — \textit{Inter-Integrated Circuit}, интерфейс микрокнтроллера.

\textit{TDP} — \textit{Thermal Dessipation Power}, рассеиваемая тепловая мощность.

\textit{PCB} — \textit{Printed Circuti Board}, печатная плата.

\newpage

\begin{center}
\textbf{ВВЕДЕНИЕ}
\end{center}

Владение системами автоматизированных инженерных расчетов явлюятся одним из важным навыков современного инженера.

В данной курсовой работе рассмотрены результаты моделирования печатной платы и анализа несколькими системами автоматизированных инженерных расчетов, использующих метод конечных элементов.

Проведен частотный и термический анализ.

Частотный анализ позволяет оценить физические характеристики печатной платы на различных рабочих частотах, что необходимо для обеспечения
стабильной работы устройства в условиях переменных нагрузок и окружающей среды.

Термический анализ, в свою очередь, позволяет выявить и устранить потенциальные проблемы с перегревом компонентов, что является критическим
аспектом для обеспечения долговечности и надежности работы устройства.
Целью курсовой работы является общетехнический анализ устройства,
моделирование его физических процессов, а также анализ полученных результатов.

\newpage

%%% Первая глава
%%% Общетехнический анализ проектируемо устройства

%%%% Анализ исходных данных
\section{ОБЩЕТЕХНИЧЕСКИЙ АНАЛИЗ ПРОЕКТИРУЕМОГО УСТРОЙСТВА}
\subsection{Анализ исходных данных}

В курсовой работе рассматривается система тестирования сервоприводов квадрокоптера.
Назначение этой системы — тестирование отдельных частей квадрокоптера,
таких как сервоприводы,
блок управления скоростью или контроллер полёта ~\cite{Elector521}.
Говоря кратко — это тестовый стенд.


Тестовый стенд выполнен в виде платы,
на которую с помощью монтажа в отверстия помещаются компоненты.
Вероятно, что такой способ мантажа выбран для того,
чтобы облегчить сборку данной схемы,
установку и смену тестируемых компонентов
или незначительного изменения схемы,
необходимого для взаиомодействия с определенными компонентами.


Поскольку в курсовой работе будет производиться расчёт тепловых
режимов будет также важно учитывать климатические факторы внешней
среды, то какие условия эксплутации при этом должны быть соблюдены
регламентирует соответсвующий стандарт — ГОСТ 15150-69
~\cite{GOST-15150-69}.

Настоящий стандарт должен применяться при проектировании изделий.  В
частности, он должен применяться при составлении технических заданий
на разработку или модернизацию изделий, а также при разработке
государственных стандартов и технических условий, устанавливающих
требования в части воздействия климатических факторов внешней среды
для группы изделий, а при отсуствии указанных групповых документов —
для отдельных видов изделий ~\cite{GOST-15150-69}.

Для конкретных типов или групп изделий виды воздействующих
климатических факторов и их номинальные значения устанавливаются в
зависимости от условий эксплуатации изделий в соответвующих
технических заданих, стандартах и технических условиях ~\cite{GOST-15150-69}.


Так как в данном случае тестовый стенд используется для того,
чтобы тестировать отдельные компоненты и их взаимодейтевие,
то логично предположить,
что в течении всего жизненного цикла изделия,
оно не будет покидать пределов лаборатории,
в которой будут проводиться испытания,
осуществляемые на данном тестовом стенде.
Основываясь на этом, можно говорить,
что климатические условия эксплуатации данного устройства соответвуют категории УХЛ 4.2 ГОСТ 15150-69.


Характеристика данной категории следующая:\\
Для эксплуатации в помещниях (объемах) с искуственно регулируемыми
климатическими условиями, например в закрытых отапливаемых или
охлаждаемых и вентелируемых производственных и других, в том числе
хорошо вентилиуруемых подземных помещниях (отсуствие воздействия
прямого соленчного излучения, атмосферных осадков, ветра, песка и пыли
наружного воздуха; отсутвие или существенное уменьшение воздействия
рассеяного солнечного излучения и конденсации влаги). Для эксплуатации
в лабораторных, капитальных и других подобного типа помещениях ~\cite{GOST-15150-69}.

Соответсвие условий работы тестового стенда определенной категории,
указанной в ГОСТ, важно по той причине,
что расчеты внешних тепловых воздействией оказываемых на данную РЭС будут производиться с использованием исходных данных о температуре,
указанных в конкретной категории ГОСТ.

%%%% Описание принципа работы анализируемого устройства
\subsection{Описание принципа работы анализируемого устройства}

Устройство представляет собой печатную плату, с установленными на ней навесным монтажом компонентами.


Сердце цепи — микроконтроллер \textit{ATmega328P},
который известен тем, что также используется в плате \textit{ArduinoUNO}.
Четыре пина \textit{GPIO} этого микрокнтроллера подключены к выводам серво-сигнала.
Серво-сигналом здесь назвам ШИМ сигнал с частотой 50 Гц.
Ширина импульса, время когда сигнал высок,
должно быть в промежутке с одной до двух милисекунд,
остальное время сигнал низок. При ширине импульса в 1,5 милисекунды сервопривод в промежуточной позиции и не вращается.
Если импульс короче, то сервопривод поворачивается в одно направление, если длиннее,
он поворачивается в противположное направление ~\cite{Elector521}.


Малогабаритные быстродействующие сервоприводы применяются в современных высокоточных системах управления подвижными объектами: рулевыми системами летательных аппаратов, автоматическими манипуляторами, роботами с подвижными элементами конструкции и др~\cite{dyakovSUBSTANTIATIONRELIABILITYSERVOMOTORS2023}.


В типичной системе контроля полета квадракоптера или дрона
пять парамеров подлежат отслеживанию:
длина четырех управляющих сервоприводами импульсов,
которые контролируют тягу, крен, тангаж и рысканье,
и напряжение питания.
В случае сложной конструкции,
например при разработке полётного контроллера,
данный тестовый стенд позволяет наблюдать получаемый сигнал ~\cite{Elector521}.



Данный тестовый стенд может работать в двух режимах,
выбираемых переключателем:
\begin{enumerate}[label={\arabic*.}]
\item Ручной режим. В нём тестовый стенд генерирует импульсы для четырех сервоприводов или полетного контроллера. Длина импульсов контролируется четырьмя потенциометрами. В этом режиме тестовый стенд обеспечивает питание сервоприводов или полетного контроллера и питание от квадрокоптера не должно быть подключено к ним. Напряжение питания тестового стенда должно быть между 7,5 или 12 Вольт.
\item Режим ввода. В нём измеряются длины имульсов поступающих от приёмника сигнала. Сигналы потом поступают на выводы, подключенные к полётному контроллеру или сервпоприводам.
  В этом режиме тестовый стенд и сервоприводы получают питание от источника питания от квадрокоптера. Получаемое питание не должно превышать 7.49 Вольт и тестовый стенд не должен быть подключен к своему истчонику питания. Также четыре канада должны быть подключены, иначе светодиод и зуммер просигналируют об ошибке.  
\end{enumerate}


Тестовый стенд измеряет длину контрольных импульсов и даёт информацию о качестве источника питания.
Данное устройство может быть подключено между приёмником дистанционного управления и полетным контроллером дрона,
или между полетным контроллером и сервоприводами.
Информацию о сигналах выводимых к сервоприводам тестовый стенд выводит на \textit{OLED} дисплей подключенный, через \textit{I2C} интерфейс.
Дисплей показывает продолжительсность импульсов графиком из четырёх кривых вмести с их численным значением в микросекундах ~\cite{Elector521}.

%%%% Анализ электрической принципиальной схемы устройства
\subsection{Анализ электрической принципиальной схемы устройства}

Поскольку статья о тестовом стэнде написана в иностранном журнале, схема электрическая принципиальная в нём выполнена не по отечественному ГОСТ.

\begin{figure}[h]
  \centering
  \includegraphics[scale = 0.60]{../img/scrot/Screenshot-2024-05-15-193101.png}
  \caption{Электрическая принципиальная схема устройства}
\end{figure}

На приложенной схеме в центре видно микроконтроллер подключенный к кристаллу с частотой колебаний 16 Мгц, который в свою очередь подключен к конденсаторам C5 и С6. Четыре пина \textit{GPIO} — PB0-PB3 подключены  к разъёму K1. Выводы для серво сигнала — PD5, PD6, PD7 и PB4 подключены к разъёму K2. Два этих раззъёма разведены таким образом, что к ним можно подключлить стандартный для сервопривода кабель ~\cite{Elector521}.

Четыре потенциометра подключеный к аналоговым входам микроконтроллера PC0-PC3. Питание подключается через делитеть напряжения из резисторов R4-R6 к аналоговому входу PC4. Отношение между суммой сопротивлений R4, R5 и сопротивлением R6 должно быть 2 к 1 соотвественно, но их абсолютные значения не критичны. Использование трёх резисторов одного значения облегчает их сортировку для точности ~\cite{Elector521}.

Для того чтобы измерять напряжение источника питания нужен аналогово цифровой преобразователь и 
опорное напряжение не относящиеся к самому напрежению питания.
В микроконтроллере уже есть опорное напряжение в 1,1 Вольт, однако это значение несколько мало. И поэтому используется источник опорного напряжения LM385-2.5 обозначенный как D2, как внешнее опорное напряжение в 2,5 Вольта. Этот элемент более точен, чем простой двухконтактный зенеровский диод ~\cite{Elector521}.

К коннектору K4 подключается \textit{OLED} дислпей по типу SSD1306.
У дисплея должен быть порт \textit{I2C}, но он будет подключаться к порту \textit{I2C} микроконтроллера, но к компонентам PD0 и PD1. Шина \textit{I2C} эмулируется программным обеспечением. Так выходит из-за того, что шина \textit{I2C} на данном микроконтроллере находится на том же пине, что и аналоговый вход PC4, который уже используется для измерения напряжения питания ~\cite{Elector521}.

Резисторы R9 и R10 это подтягивающие резисторы для шины \textit{I2C}.

Ползунковый переключатель S1 типа \textit{DPDT} используется для выбора режима работы тестового стенда. В ручном режиме переключатель соединяет напряжения питания номиналом 5 Вольт и коннекторы сервоприводов. В режиме ввода он предовтращает неправильное включение напряжений ~\cite{Elector521}.

Возможно, что решение в части подключение дисплея к данному тестовому
стенду и не является самым элегантным. Однако сама по себе схема
достаточно проста,состоит из небольшого числа комопонентов, но тем не
менее может работать в нескольких режимах и схемах включения, что выглядит как положительные черты относительно того в каких вариантах использования оказывается устройство.

%%%% Определение параметров
%%%% воздействующих дестабилизирующих факторов и
%%%% протекающих физических процессов

% Приплести сюда ГОСТ на устройство
\subsection{Определение параметров воздействующих дестабилизирующих факторов протекающих физических процессов}


РЭС эксплуатируются в условиях воздействия на них целого ряда
систематических и случайных факторов. Систематические факторы определяют рабочие функции аппаратуры и могут быть учтены в процессе разработки,
а случайные факторы по своему характеру и времени воздействия проявляются случайным образом, в связи с чем учет их влияния в процессе разработки достаточно затруднен ~\cite{Alexeev2013}.

Факторы делятся на объективные, не зависящие от человека, и субъективными, то есть зависящими от действий конструктора или использующего данный тестовый стенд оператора.

По характеру воздействия на РЭС объективные факторы разделяются на климатические, механические и другие.

Параметры климатических факторов принимают значения в соотвестиве с учетом УХЛ 4.2 ГОСТ 15150-69.

Параметры механических факторов же принимают значения, выбранные разработчиком.

Наиболее вероятным механическим фактором для данного устройства будет воздействие вибрации, в следствии работы подключаемых сервоприводов.

Во всей схеме практически не встречаются компоненты с большим тепловыделением. Так например найти коэффициент \textit{TDP} для микроконтроллера не предстовляется возомжным, потому что среди производителй микроконтроллеров, в отличие от производителей микропроцессоров не принято не то что рассчитывать этот коэффициент рассеивания теплоты, а даже закладывать в этот компоенент схемы возможности какого-либо его нагрева.

Однако это ни в коем случае не означает то, что ни один из элементов не будет нагреваться. Напротив при работе любого электронного прибора какая-то часть потребляемой мощности рассеивается как тепло.
Но если рассматривать отношение мощности потребляемой и рассеиваемой как тепло, то можно понять, что самым простым оно будет у резисторов.

По видимому, можно взять информацию о мощности резистора и принять её за мощность рассеиваемую в виде теплоты ~\cite{HeatDissipatedResistors}.

Примерно таким же образом было сделано допущение о схожести в вопросе рассеиваемой мощности между резисторами и потенциометрами.

Исходя из всего вышеописанного был сделан вывод о необходимости проведения частотного анализа и термического анализа.

\newpage

%%% Вторая глава по разделам


\section{Моделирование физических процессов, воздействующих на устройство}
\subsection{Обоснование выбора прикладного программного обеспечения для моделирования физических процессов}

Программ для проведения инженерных расчетов не так уж много. И итак ограниченный выбор пал на три программы: \textit{SOLIDWORKS SIMULATION},
\textit{ANSYS}, \textit{COMSOL Multiphysics}. Дело в том, что все три эти программы, чаще всего и используются для МКЭ расчетов. Хотя, это по большей части относится к \textit{COMSOL Multiphysics} и \textit{ANSYS}, а \textit{SOLIDWORKS} удобен и сам по себе как средство для конвертации, работы и упрощения с трёхмерными параметрическими моделями.
Возможность моделировать в нём можно считать приятным дополненим к его функционалу, которое тем не менее видится как обязательное к использованию в этой работе.

Популярность таких программ как \textit{COMSOL Multiphysics} и \textit{ANSYS} в кругах инженеров позволяет рассчитывать на наличие большого количетства обучающих материалов по ним, в том числе созданных не только самой компанией осуществляющей разработку программы.

Кроме того обе программы сами по себе обладают огромным набором возможностей для проведения симуляции и зачастую имеют некоторый аналог интерфейса трёхмерных параметрических САПР типа \textit{SOLIDWORKS}, который позволяет интерактивно взаимодествовать с геометрией модели, материалом и сеткой модели.

Также особенно хочется отметить доступность и качество документации доступной на официальном сайте \textit{COMSOL}.

Совокупнсот вышеописанных качеств делает эти программы достойным выбором для проведения исследования.

\subsection{Разработка плана моделирования физических процессов}

Под моделированием можно понимать метод исследования, оценок характеристик сложных сечений используемый для принятия решения в различных сферах инженерной деятельности.

Под моделью следует понимать объект заместитель, который в определенных условиях может заменить объект оригинала воспроизводя интересующее исследователя свойство оригинала. Чем более подробное описание модели требуется, тем более длительным и затратным окажется вычислительный
процесс.

Основными целями моделирования является: прогноз (главная цель моделирования), оптимизация и анализ чувствительности.

Следовательно, план моделирование физических процессов выглядит так:
\begin{enumerate}[label={\arabic*.}]
\item Построение нескольких варинтов компоновок исследоваемого тестового стенда.
\item Экспорт получившихся печатных плат в виде трёхмерных моделий пригодных для трёхмерного параметрического моделирования и расчетов методом конечных элементов.
\item Проведение инженерных анализов с использованием различных CAE программ.
\item Сравнение результатов анализиа, полученных из программ.
\item Определние адекватности полученных результатов.  
\end{enumerate}



В случае получения неадекватных результатов моделирование повторяется с первого пункта.

Из этого следует что для более детального анализа необходимо
смоделировать различные варианты компоновки тестового стенда, а также
смоделировать и проанализировать, для каждого варианта исполнения,
результаты физических процессов. Моделирование будет проводится до тех
пор, пока не будут получены адекватные результаты.


\subsection{Методика построения трехмерной модели исследуемого устройства}

Создание модели печатной платы было начато в САПР \textit{Altium Designer}.
В нём была создана библиотека УГО принципиальная схема устройства.
Затем, по логике создания моделей в САПР печатных плат подобных \textit{Altium Designer} должны были быть созданы библиотеки посадочных мест, соедиение их с УГО размещенных на принципиальной схеме.
После чего надо было бы разнести посадочные места на печатной плате и произвести трассировку.

Однако я поступил не так.
Из-за своего изначального предпочтения к использовонию программ с открытым исходным кодом было принято решение использовать САПР печатных плат \textit{KiCAD}.

В тот момент процедура создания печатных плат в \textit{KiCAD}, казалась проще, чем в \textit{Alitum Designer}. Всё дело в том, что у \textit{KiCAD} существует своя библиотека посадочных мест с трёхмерными моделями соотвествующих компонентов в формате \textit{STEP} и \textit{WRL}. Также на \textit{Github} была найдена готовая библиотека УГО соотвествующая ГОСТ.
В \textit{KiCAD} нет предустановленного встроенного автотрассировщика и потому разводка платы производилась вручную,
несмотря на то что можно было бы установить автотрассировщик \textit{freerouting}.

На этом этапе казалось, что выбор \textit{KiCAD} в качестве САПР печатных плат всё только упрощает, поскольку теперь было достаточно:
\begin{enumerate}[label={\arabic*.}]
\item Создать схему электрическую принципиальную, используя готовую библиотеку УГО.
\item На схеме установить соотвествие между УГО и посадочными местами, у которых есть свои трёхмерные модели.
\item Импортировать получившиеся связи непосредственно в документ, в котором осуществляется разводка печатной платы.
\item Рапределить компоненты в группы и провести дорожки наиболее эффективных способом.
\end{enumerate}

Это и было сделано. На тот момент подозрений о том, что это приведёт к каким-либо проблемам не было.

\begin{figure}[H]
  \centering
  \includegraphics[scale = 0.20]{../img/scrot/Screenshot-2024-05-02-173914.png}
 \caption{Скриншот окна САПР \textit{KiCAD} в котором скрыты цепи относящиеся к питанию и компоненты распределены по группам.}
\end{figure}

Была произведена разводка первой печатной платы:
сначала скрыты цепи связанные с питанием, затем компоненты были объеденены в группы, после чего расставлены по контуру печатной платы и соедены дорожками.

Чтобы заново не производить разводку, а если быть предельно точным,
соединение компонентов было сделать две последующие компоновки печатной платы следующим способом:


В случае второй компоновки просто заменить расположение некоторых посадочных мест, таким образом, чтобы они оказались на другой стороне печатной платы, но подали в нужные отверстия печатной платы. В основном это относилось к резисторам, которые в последствии были использованы, как элементы выделяющие тепло в термическом анализе.


В случае третьей платы были заменены некоторые посадочные места, на минитюаризированные аналоги, в описании которых при этом значился тот же самый шаг (\textit{pitch}) между отверстиями.

\begin{figure}[H]
  \centering
  \includegraphics[scale = 0.20]{../img/scrot/Screenshot-2024-05-07-112847.png}
 \caption{Скриншот окон просмотрщика трёхмерных моделей САПР \textit{KiCAD}}
\end{figure}


Полученные варианты печатной платы необходимо экспортировать в \textit{SOLIDWORKS} для проведения дальньшей симуляции. И вот здесь возникли проблемы с получившийся моделью.

\begin{figure}[H]
  \centering
  \includegraphics[scale = 0.4]{../img/scrot/Screenshot-2024-05-06-191242.png}
  \caption{Скриншот меню экспорта печатной платы в САПР \textit{KiCAD} }
\end{figure}

В первую очередь была осуществлена попытка экспортировать печатную плату из \textit{KiCAD} в формате \textit{STEP}. Конечно же модель успешно экспортировалась. Проблема была в другом: полученная модель была излишне подробна. Расчет такой модели в \textit{SOLIDWORKS Simulation} мог занимать по пять минут для частотного анализа, но, что более критично, часто приводил к зависанию или падению \textit{SOLIDWORKS}.

Для того, чтобы решить эту проблему было предпринята попытка экспортировать в формате \textit{IDFv3}. Это решение было принято из-за того, что подразумевалось,
что при импорте полученного \textit{IDF} файла в \textit{SOLIDWORKS} будет автоматически использовано дополнение \textit{Circuit WORKS}, которое в случае импорта из \textit{Аltium Designer} упрощало модель таким образом, что вместо точных трехмерных моделей компонентов получались схожие, но значительно упрощенные трёхмерные геометрические фигуры.


\begin{figure}[H]
  \centering
  \includegraphics[scale=0.2]{../img/scrot/Screenshot-2024-05-06-175849.png}
  \caption{Скриншот импортированного IDF печатной платы в САПР \textit{SOLIDWORKS}}
\end{figure}

Однако и эта попытка кончилась провалом. Импортированная плата не имела вообще никаких компонентов и была упрощена черезмерно.

Было понятно, что нужно будет переделывать в \textit{Altium Designer} какую-то часть проекта, чтобы получить полноценную трёхмерную модель печатной платы.

Однако совсем не хотелось терять проделанную работу, поэтому было установлено расширение для \textit{Altium Designer} под названием \textit{KiCAD Importer}, которое предоставляло мастер импорта проектов из \textit{KiCAD}.

\begin{figure}[H]
  \centering
  \includegraphics[scale=0.2]{../img/scrot/Screenshot-2024-05-08-110214.png}
  \caption{Скриншот трёхмерной модели имопрированной из \textit{KiCAD} печатной платы в \textit{Altium Designer}}

\end{figure}

В результате импорта в проекте \textit{Altium} появились соотвественно три документа печатных плат.

После этого модели были экспортированы из \textit{Altium} в формате \textit{IDF}. Затем был осуществлен импорт этих моделей в \textit{SOLIDWORKS}. В отличие от моделей получившихся при предыдущей попытке экспорта теперь была сохранена геометрия платы и остались контуры (\textit{Sketch}) от элементов расположенных на ней. В последствии инструментом \textit{Extrude} на вкладке \textit{Analysis Prepartion} было осуществлено выдавливание фигур, схожих по объёму с трехмерным моделями печатных плат.

\begin{figure}[H]
  \centering
  \includegraphics[scale=0.3]{../img/scrot/Screenshot-2024-05-08-132748.png}
  \caption{Скриншот только импортированной в \textit{SOLIDWORKS} печатной платы в формате IDF из \textit{Altium Designer}. Видны контуры посадочных мест.}

\end{figure}


Изначально зная о том, сколько работы будет потрачено, на всего лишь, экспорт трёхмерной модели печатной платы я бы не стал использовать для этого САПР \textit{KiCAD} даже несмотря на возможное удобство в непоредственно проектировании.

Из этого можно сделать вывод, что при выполнение подобных проектов, при выбьоре определенного САПР стоит руководствоваться совместимостью его с другими программами.
Или хотя бы заренее проверять на каком-то миниатюрном проекте-приемере возможность перенесения модели из одной программы в другую.

Что же касается непосредственно проектирвония, то, возможно, было опрометчиво отказываться от использования монтажных отверстий в плате.



\subsection{Моделирование механических процессов, проетекающих в электронном модуле и устройстве в целом}

Результаты частотного анализа в \textit{SOLIDWORKS Simulation} первого варианта компоновки ПП представлены на рисунке 8.

\begin{figure}[H]
  \centering
  \includegraphics[scale=0.3]{../img/sst-1/freq/top_view/sst-sst-1-freq-Amplitude-Amplitude1.jpg}
  \caption{Результаты частотного анализа в \textit{SOLIDWORKS SIMULATION} первого варианта компоновки ПП.}
\end{figure}


На рисунке 9 представлены результаты частотного анализа в \textit{SOLIDWORKS Simulation} второго варианта компоновки печатной платы.

\begin{figure}[H]
  \centering
  \includegraphics[scale = 0.3]{../img/sst-2/freq/sst_v2-sst-2-freq-2-Amplitude-Amplitude1.jpg}
  \caption{Результаты частотного анализа в \textit{SOLIDWORKS SIMULATION} второго варианта компоновки ПП.}

\end{figure}

На рисунке 10 представлены результаты частотного анализа в
\textit{SOLIDWORKS Simulation} третьего варианта компоновки печатной платы.


\begin{figure}[H]
  \centering
\includegraphics[scale=0.3]{../img/sst-3/freq/sst_v3-Freq-3-Amplitude-Amplitude1.jpg}
\caption{Результаты частотного анализа в \textit{SOLIDWORKS Simulation} третьего варианта компоновки ПП.}
\end{figure}

На рисунке 11 представлены результаты частотного анализа в
\textit{COMSOL Multiphysics} первого варианта компоновки печатной платы.

\begin{figure}[H]
  \centering
  \includegraphics[scale=0.3]{../img/sst-1/freq/comsol.png}
  \caption{Результаты частотного анализа в \textit{COMSOL Multiphysics} первого варианта компоновки}
\end{figure}

На рисунке 12 представлены результаты частотного анализа в \textit{COMSOL Multiphysics} второго вараинта компоновки печатной платы.

\begin{figure}[H]
  \centering
  \includegraphics[scale=0.3]{../img/scrot/Screenshot-2024-05-16-014808.png}
  \caption{Результаты частотного анализа в \textit{COMSOL Multiphysics}
    второго варианта компоновки}
\end{figure}

На рисунке 13 представлены результаты частотного анализа в \textit{COMSOL Multiphysics} третьего варианта компоновки печатной платы.

\begin{figure}[H]
  \centering
  \includegraphics[scale=0.3]{../img/sst-3/freq/comsol.png}
  \caption{Результаты частотного анализа в \textit{COMSOL Multiphysics}
    третьего варианта компоновки}
\end{figure}


\subsection{Моделирование тепловых процессов, протекающих в электронном модуле и устройстве в целом}

Результаты теплового анализа для первого варианта компоновки ПП
в \textit{SOLIDWORKS Simulation} представлены на рисунке 14.

\begin{figure}[h]
  \centering
  \includegraphics[scale=0.3]{../img/sst-1/thermal/top_view.png}
  \caption{Результаты теплового анализа в \textit{SOLIDWORKS Simulation}
    для первого варианта компоновки}
\end{figure}

Результаты теплового анализа для второго варианта компоновки ПП в
\textit{SOLIDWORKS Simulation} представлены на рисунке 15.

\begin{figure}[h]
  \centering
  \includegraphics[scale=0.3]{../img/sst-2/thermal/sst_v2-sst-2-thermal-Thermal-Thermal3.jpg}
  \caption{Результаты теплового аналиаза в \textit{SOLIDWORKS Simulation}
    для второго варианта компновки}
\end{figure}

Результаты теплового анализа для третьего варианта компоновки ПП в
\textit{SOLIDWORKS Simulation} представлены на рисунке 16.
\begin{figure}[h]
  \centering
  \includegraphics[scale=0.3]{../img/sst-3/thermal/sst_v3-sst-3-thermal-Thermal-Thermal3.jpg}
  \caption{Результаты теплового аналиаза в \textit{SOLIDWORKS Simulation}
    для второго варианта компновки}
\end{figure}

Результаты тепловго анализа для первого варианта компоновки ПП в
\textit{COMSOL Multiphysics} представлены на рисунке 17.
\begin{figure}[h]
  \centering
  \includegraphics[scale=0.5]{../img/scrot/Screenshot-2024-05-16-022638.png}
  \caption{Результаты теплового анализа в \textit{COMSOL Multhiphysics}
    для первого варианта компоновки}
\end{figure}

Результаты теплвого анализа для второго варианта компоновки ПП в
\textit{COMSOL Multiphysics} представлены на рисунке 18.

\begin{figure}[h]
  \centering
  \includegraphics[scale=0.5]{../img/scrot/Screenshot-2024-05-16-023133.png}
  \caption{Результаты теплвого анализа в \textit{COMSOL Multiphysics}
  для второго варианта компоновки}

\end{figure}

Результаты тепловго анализа для третьего варианта комопновки ПП в
\textit{COMSOL Multiphysics} представлены на рисунке 19.
\begin{figure}[H]
  \centering
  \includegraphics[scale=0.5]{../img/scrot/Screenshot-2024-05-16-024105.png}
  \caption{Результаты теплвого анализа в \textit{COMSOL Multiphysics}
  для третьего варианта компоновки}

\end{figure}





\newpage
\section{Анализ полученных результатов моделирования}
\subsection{Обработка, анализ и интепретация данных проведенного моделирования механических процессов}

\begin{table}[h]
  \centering
  \caption{Результаты частотного анализа в \textit{SolidWorks}}
\begin{tabular}{|c|c|}
\hline
  Вариант компоновки ПП & Значение собственной частоты \\
  1 & 10,1Гц \\
  2 & 7,3 Гц \\
  3 & 7,9 Гц \\
\hline
\end{tabular}
\end{table}

Самое большое значение частоты у печатной платы выполненненой в первой компоновке. Однако можно заметить, что во всех трёх случаях частота досточно мала. 10 Гц это 10 колебаний платы в секунду, что несоотвевует уровню требуему данной печатной плате.

Однако можно считать, что в значительной степени такой результат вышел потому, что плата крепилась только за одну поверхность и была очень тонкой, так как такой вышла модель в результате экспорта.

\begin{table}[h]
  \centering
  \caption{Результаты частотного анализа в \textit{COMSOL Multiphysics}}
\begin{tabular}{|c|c|}
\hline
  Вариант компоновки ПП & Значение собственной частоты \\
  1 & 0.1 Гц \\
  2 & 13509 Гц \\
  3 & 14 Гц \\
\hline
\end{tabular}
\end{table}

Результаты анализа в \textit{COMSOL Multiphysics} так разнятся, что сделать какой-либо вывод, а не исключить данные из анализа сложно.
Но судя по всем именно значение 13509 Гц ближе всего к тому значению, которое должно было получиться при анализе печатной платы в независомости от компоновки.


\subsection{Обработка, анализ и интепретация данных проведенного моделирования телповых процессов}

\begin{table}[h]
  \centering
  \caption{Результаты теплового анализа в \textit{SolidWorks}}
\begin{tabular}{|c|c|}
\hline
  Вариант компоновки ПП & Максимальное начение температуры в Цельсиях \\
  1 &  150 \\
  2 &  145 \\
  3 &  143 \\
\hline
\end{tabular}
\end{table}

Самое большое значение температуры у печатной платы в первой компановке.
По видимому оба решения с разнесеним компонентов по разные стороны платы
и заменой тепловыделяющих компонентов на меньшие по размеру выглядят приемлимыми для обеспечения теплового режима ПП.

Но только теплового режима  печатной платы.

\begin{table}[h]
  \centering
  \caption{Результаты теплового анализа в \textit{COMSOL Multiphysics}}
\begin{tabular}{|c|c|}
\hline
  Вариант компоновки ПП & Значение максимальной температуры в Цельсиях \\
  1 & 41.3 \\
  2 & 47,5 \\
  3 & 47 \\
\hline
\end{tabular}
\end{table}

Моделирование в \textit{COMSOL Multiphysics} показало ровно противоположный результат. При этом в качестве геометрии была импортирована модель из \textit{SOLIDWORKS} с расширением \textit{.SLPDRT}

По видимому несовпадение результатов вызвано различным методом решения применямым в данных программах.

Однако я склонен больше доверять \textit{COMSOL}, как программер специально предназначенных для таких рассчетов.

По полученным результатам и графикам можно сделать вывод,
что при проектировании печатной платы, точнее при, так называемой разводке, решение о том как расставлять компоненты можно принимать
не только основываясь на том как близко должен быть элемент с точки зрения схемотехники, но также учитывать, что рассеивающие тепло в значительной степени компоненты будут нагреваться до больших температур в тех случаях, когда будут расставлены рядом.

\newpage

\begin{center}
\textbf{Заключение}
\end{center}

В результате выполнения курсового проекта проведен общетехнический
анализ проектируемого устройства, который включает: анализ исходных данных, описание принципа работы анализируемого устройства, анализ электрической принципиальной схемы устройства, определение параметров воздействующих дестабилизирующих факторов и протекающих физических процес-
лсов для последующего моделирования.

Проведено моделирование физических процессов, воздействующих на
устройство, а именно: обоснование выбора прикладного программного обеспечения для моделирования физических процессов, разработан план моделирования физических процессов, разработана методика построения трехмерной модели исследуемого устройства, проведено моделирование механических
процессов, протекающих в электронном модуле и устройстве в целом, и проведено моделирование тепловых процессов, протекающих в электронном модуле и устройстве в целом.
Основываясь на вышеперечисленном, был проведен анализ полученных
результатов, а также дана краткая оценка.

Таким образом была предпринята попытка реализовать все поставленные цели и задачи на курсовой проект. Однако с уверенностью можно заключить, что эта попытка не была удачной. Большую роль сыграли проблемы с экспортом модели трехмерной модели, случившиеся в результате перехода на другую систему автоматизированного проектирования печатной платы.

Остается только сделать выводы и не допускать ошибок, подобных допущенным при выполнении данного курсового проекта.
  \newpage



% \bibligoraphystyle{5sem}
% \renewcommand{\bibsection}{{Cписок использованных источников}}
\renewcommand{\refname}{\textbf{Cписок использованных источников}}
\DeclareFieldFormat{url}{Режим доступа\addcolon\space\url{#1}}
\DeclareFieldFormat{title}{{#1}}
\DeclareFieldFormat{labelnumberwidth}{[{#1}]\adddot  }
\printbibliography

\newpage
\begin{center}
\textbf{Приложение А}\\
\textbf{Обязательное}\\
\textbf{Параметры компонентов}
\end{center}

Техническое описание ПП:
\begin{enumerate}[label={\arabic*.}]
\item Соотношение сторон 1:1.
\item Габариты ПП 75 на 75 мм.
\item Толщина ПП 0,8 мм.
\end{enumerate}

Данные необходимые для тепловго анализа:
\begin{enumerate}[label={\arabic*.}]
\item Температура окружающей среды 315 К.
\item Коэффициент конвекции — 25 Вт/К * м².
\end{enumerate}

\newpage


\begin{center}
\textbf{Приложение Б}\\
\textbf{Обязательное}\\
\textbf{Отчет о проверке на заимствования в системе «Антиплагиат»}

\begin{figure}[H]
  \centering
  \includegraphics[scale=0.7]{../img/antiplagiat.pdf}
  \caption{Результат проверки на заимстования в системе «Антиплагиат»}
\end{figure}
\end{center}

\end{document}
