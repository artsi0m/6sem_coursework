\section{Моделирование физических процессов, воздействующих на устройство}
\subsection{Обоснование выбора прикладного программного обеспечения для моделирования физических процессов}

Программ для проведения инженерных расчетов не так уж много. И итак ограниченный выбор пал на три программы: \textit{SOLIDWORKS SIMULATION},
\textit{ANSYS}, \textit{COMSOL Multiphysics}. Дело в том, что все три эти программы, чаще всего и используются для МКЭ расчетов. Хотя, это по большей части относится к \textit{COMSOL Multiphysics} и \textit{ANSYS}, а \textit{SOLIDWORKS} удобен и сам по себе как средство для конвертации, работы и упрощения с трёхмерными параметрическими моделями.
Возможность моделировать в нём можно считать приятным дополненим к его функционалу, которое тем не менее видится как обязательное к использованию в этой работе.

Популярность таких программ как \textit{COMSOL Multiphysics} и \textit{ANSYS} в кругах инженеров позволяет рассчитывать на наличие большого количетства обучающих материалов по ним, в том числе созданных не только самой компанией осуществляющей разработку программы.

Кроме того обе программы сами по себе обладают огромным набором возможностей для проведения симуляции и зачастую имеют некоторый аналог интерфейса трёхмерных параметрических САПР типа \textit{SOLIDWORKS}, который позволяет интерактивно взаимодествовать с геометрией модели, материалом и сеткой модели.

Также особенно хочется отметить доступность и качество документации доступной на официальном сайте \textit{COMSOL}.

Совокупнсот вышеописанных качеств делает эти программы достойным выбором для проведения исследования.