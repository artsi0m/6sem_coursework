\subsection{Разработка плана моделирования физических процессов}

Под моделированием можно понимать метод исследования, оценок характеристик сложных сечений используемый для принятия решения в различных сферах инженерной деятельности.

Под моделью следует понимать объект заместитель, который в определенных условиях может заменить объект оригинала воспроизводя интересующее исследователя свойство оригинала. Чем более подробное описание модели требуется, тем более длительным и затратным окажется вычислительный
процесс.

Основными целями моделирования является: прогноз (главная цель моделирования), оптимизация и анализ чувствительности.

Следовательно, план моделирование физических процессов выглядит так:
\begin{enumerate}[label={\arabic*.}]
\item Построение нескольких варинтов компоновок исследоваемого тестового стенда.
\item Экспорт получившихся печатных плат в виде трёхмерных моделий пригодных для трёхмерного параметрического моделирования и расчетов методом конечных элементов.
\item Проведение инженерных анализов с использованием различных CAE программ.
\item Сравнение результатов анализиа, полученных из программ.
\item Определние адекватности полученных результатов.  
\end{enumerate}



В случае получения неадекватных результатов моделирование повторяется с первого пункта.

Из этого следует что для более детального анализа необходимо
смоделировать различные варианты компоновки тестового стенда, а также
смоделировать и проанализировать, для каждого варианта исполнения,
результаты физических процессов. Моделирование будет проводится до тех
пор, пока не будут получены адекватные результаты.
